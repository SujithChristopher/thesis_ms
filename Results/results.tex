\documentclass[12pt, twoside]{report}
\usepackage{layout}
\usepackage[utf8]{inputenc}
\usepackage{graphicx}
\usepackage{amsmath}
% \usepackage{fontspec}
\usepackage[a4paper, width=150mm, top=25mm, bottom=25mm, left = 4.0132cm, right = 25mm]{geometry}

\usepackage{setspace}
\usepackage{titlesec}

\titleformat*{\section}{\Huge\bfseries}
\titleformat*{\subsection}{\Large\bfseries}
\titleformat*{\subsubsection}{\large\bfseries}
\titleformat*{\paragraph}{\large\bfseries}
\titleformat*{\subparagraph}{\large\bfseries}


\graphicspath{ {./src/} }



\begin{document}
% new section with size 24

\chapter{Results}

\section{Investigation of the kinematics of the robot}

This general expression is taken from the book
"Modern robotics mecanics planning and control" by Kevin M Lynch, Frank C Park,
page no 512, please refer this section for more details.

\section{Three wheel omnidirectional robot}


\begin{equation}
    h_1(0) \mathcal{V}_b = \frac{1}{r_i} \left[\begin{array}{cc}
            1 & \tan\gamma_i \\
        \end{array}\right]
    \left[\begin{array}{cc}
            \cos \beta_i  & \sin \beta_i \\
            -\sin \beta_i & \cos \beta_i \\
        \end{array}\right]
    \begin{bmatrix}
        -y_i & 1 & 0 \\
        x_i  & 0 & 1 \\
    \end{bmatrix} \mathcal{V}_b
\end{equation}

\begin{equation*}
    \gamma_1 = 0, \beta_1 = 0
\end{equation*}

\begin{equation}
    = \frac{1}{r_1} \left[\begin{array}{cc}
            1 & 0 \\
        \end{array}\right]
    \left[\begin{array}{cc}
            \cos 0  & \sin 0 \\
            -\sin 0 & \cos 0 \\
        \end{array}\right]
    \begin{bmatrix}
        -y_1 & 1 & 0 \\
        x_1  & 0 & 1 \\
    \end{bmatrix}
    \mathcal{V}_b
\end{equation}

\begin{equation}
    = \frac{1}{r_1} \left[\begin{array}{cc}
            1 & 0 \\
        \end{array}\right]
    \begin{bmatrix}
        1 & 0 \\
        0 & 1 \\
    \end{bmatrix}
    \begin{bmatrix}
        -y_1 & 1 & 0 \\
        x_1  & 0 & 1 \\
    \end{bmatrix}
    \mathcal{V}_b
\end{equation}
\begin{equation}
    = \frac{1}{r_1}
    \begin{bmatrix}
        -y_1 & 1 & 0 \\
    \end{bmatrix}
    \mathcal{V}_b
\end{equation}

\begin{equation}
    r_i = r_1 = r_2 = r_3 = r
\end{equation}

\begin{equation}
    h_1(0) \mathcal{V}_b = \frac{1}{r}
    \begin{bmatrix}
        -y_1 & 1 & 0 \\
    \end{bmatrix}
    \mathcal{V}_b
\end{equation}

% for 2nd wheel
% bold text here

Similarly for the second wheel

\begin{equation}
    h_2(0) \mathcal{V}_b = \frac{1}{r}
    \begin{bmatrix}
        \frac{1}{2}y_2 - \frac{\sqrt{3}}{2}x_2 & \frac{-1}{2} & \frac{- \sqrt{3}}{2} \\
    \end{bmatrix}
    \mathcal{V}_b
\end{equation}

% for 3rd wheel
% bold text here

Similarly for the third wheel

\begin{equation}
    h_3(0) \mathcal{V}_b = \frac{1}{r}
    \begin{bmatrix}
        \frac{1}{2}y_3 + \frac{\sqrt{3}}{2}x_3 & \frac{-1}{2} & \frac{ \sqrt{3}}{2} \\
    \end{bmatrix}
    \mathcal{V}_b
\end{equation}


We know that

\begin{equation}
    y_2 = y_3,  x_2 = -x_3
\end{equation}

The final expression for the three wheel omnidirectional robot is

\begin{equation}
    H(0)\mathcal{V}_b = \frac{1}{r}
    \begin{bmatrix}
        -y_1                                   & 1            & 0                    \\
        \frac{1}{2}y_2 - \frac{\sqrt{3}}{2}x_2 & \frac{-1}{2} & \frac{- \sqrt{3}}{2} \\
        \frac{1}{2}y_3 + \frac{\sqrt{3}}{2}x_3 & \frac{-1}{2} & \frac{ \sqrt{3}}{2}  \\
    \end{bmatrix}
    \begin{bmatrix}
        \omega_{bz}   \\
        \upsilon_{bx} \\
        \upsilon_{by} \\
    \end{bmatrix}
\end{equation}

To find
$
    \begin{bmatrix}
        \omega_{bz}   \\
        \upsilon_{bx} \\
        \upsilon_{by} \\
    \end{bmatrix}
$\\

% insert symbol
Lets insert the symbol for the above expression
\begin{align}
    \mathcal{A}  =
    \begin{bmatrix}
        -y_1                                   & 1            & 0                    \\
        \frac{1}{2}y_2 - \frac{\sqrt{3}}{2}x_2 & \frac{-1}{2} & \frac{- \sqrt{3}}{2} \\
        \frac{1}{2}y_3 + \frac{\sqrt{3}}{2}x_3 & \frac{-1}{2} & \frac{ \sqrt{3}}{2}  \\
    \end{bmatrix}
\end{align}

Solving the above will lead to the following expression

\begin{align}
    \mathcal{A} =
    \begin{bmatrix}
        -d & 1            & 0                    \\
        -d & \frac{-1}{2} & \frac{- \sqrt{3}}{2} \\
        -d & \frac{-1}{2} & \frac{\sqrt{3}}{2}   \\
    \end{bmatrix}
\end{align}

To find
$
    \begin{bmatrix}
        \omega_{bz}   \\
        \upsilon_{bx} \\
        \upsilon_{by} \\
    \end{bmatrix}
$ = $r\mathcal{A}^{-1} u$


\newpage

\section*{ArUco marker detection}

An ArUco marker is a square-shaped marker that has a wide black border and an inner
binary matrix that determines its identifier as shown in FIG. ArUco markers are commonly
used in computer vision applications such as robot navigation and augmented reality.
The ArUco marker has a fixed dimensions which helps the program to estimate its position and orientation, in real-world frame.
\end{document}