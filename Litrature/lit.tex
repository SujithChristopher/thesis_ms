\documentclass[12pt, twoside]{report}
\usepackage{layout}
\usepackage[utf8]{inputenc}
\usepackage{graphicx}
\usepackage{amsmath}
\usepackage[a4paper, width=150mm, top=25mm, bottom=25mm]{geometry}


\graphicspath{ {./src/} }

\begin{document}
\chapter{Litrature survey}

Kuan Cha et al, have developed a novel virtual reality rehabilitation 
system (VRRS) for upper-limb rehabilitation. To the motion of both upper 
limbs, integrate fine finger motion and the range of motion of the entire 
arm, and map the motion to an avatar. This uses a two-camera setup, 
to find the position of the arm on the table. This system improved 
the user experience of embodiment and effectively improved the effects 
of upper limb rehabilitation in stroke patients. But however, 
this particular setup needs to be calibrated frequently, and requires a 
lot of setup time.\\

Khadijeh Moulaei et al, investigated the role of robots in upper limb 
disability improvement and rehabilitation. They found that most robots 
were used to rehabilitate stroke patients and about 60.52{\%} of the studies 
used games and virtual reality to rehabilitate upper limb disabilities using 
robots. The most important outcomes of upper limb rehabilitation robots 
were “Improvement in musculoskeletal functions”, “No adverse effect on
 patients”, and "Safe and reliable treatment". \\

Zhang et al. devised a deep learning method based on the Single Shot 
Detector (SSD) approach to identify the fiducial marker even in the 
presence of motion blur. This model was trained using a combination 
of synthetic and real-world datasets, which mimicked various scenarios 
such as motion blur, low light, and minor occlusion. 
Their research yielded a 100 percent success rate in marker detection 
and a 98 percent accuracy in pose estimation. \\

The ability to successfully detect ArUco markers hinges on several factors, 
such as the resolution of the camera and the speed of its aperture. 
Cameras that are less expensive may not be ideal for estimating the 
pose of ArUco markers, as the swift movement of the markers can result 
in motion blur when the aperture setting is slow. Various solutions to 
this issue have been explored by numerous researchers, 
including Garrido-Jurado et al., 2014 and Zhang et al., 2021. \\

\end{document}