\documentclass[12pt, twoside]{report}
\usepackage{layout}
\usepackage[utf8]{inputenc}
\usepackage{graphicx}
\usepackage{amsmath}
\usepackage[a4paper, width=150mm, top=25mm, bottom=25mm]{geometry}
\usepackage{biblatex}
\addbibresource{bibs.bib}


\graphicspath{ {./src/} }

\begin{document}
\chapter{Introduction}
\section{Stroke}
Recent studies show us that cumulative incidence of stroke in India ranged from 105 to 152/100,000 persons per year, and the crude prevalence of stroke ranged from 44.29 to 559/100,000 persons in different parts of the country during the past decade. 
\parencite{kamalakannan_incidence_nodate} The most common type of stroke in India is ischemic stroke, 
which accounts for about 80\% of all strokes. \\

Ischemic stroke occurs when a blood clot blocks an artery in the brain.
Hemorrhagic stroke, which occurs when a blood vessel in the brain ruptures, 
accounts for about 20\% 
of all strokes in India. 
The percentage of patients who have upper limb impairment in stroke is 
estimated to be between 50\% and 80\%. 
This means that about half to two-thirds of stroke survivors will experience 
some degree of weakness, stiffness, or coordination problems 
in their affected arm or hand \parencite{parker_loss_2009, kamalakannan_incidence_nodate}.\\

\section{Challenge in stroke rehabilitation}

Limited and inadequate access to healthcare, 
coupled with minimal in-clinic therapy time, can pose significant obstacles 
to recovery \parencite{housman_randomized_2009}. In-home rehabilitation can serve as an effective solution 
to these challenges. This includes training that simulates simple arm 
movements, such as peg tests and stacking cups. One particularly effective 
form of training involves arm-skateboard exercises. In this method, 
the patient places their affected arm on a simple skateboard on a table 
and practices reaching exercises. As the weight of the arm is supported 
by the table \parencite{housman_randomized_2009,conroy_effect_2011}, 
the patient can move their affected arm more effectively.


While arm-skateboard exercises are a popular and cost-effective method 
for improving upper-limb mobility \parencite{chanubol_randomized_2012,sanchez_automating_2006}, they often lack regular feedback 
for the patient, which can lead to boredom and poor adherence to therapy. 
In this study, we have developed a simple and inexpensive arm-skateboard
 that provides real-time feedback to patients during their therapy 
 sessions.


\newpage
\section{Aim of the study}

While arm-skateboard exercises are a popular and cost-effective method for 
improving upper-limb mobility \parencite{chanubol_randomized_2012,sanchez_automating_2006}, they often lack regular feedback for 
the patient, which can lead to boredom and poor adherence to therapy. 
In this study, we have developed a simple and inexpensive arm-skateboard 
that provides real-time feedback to patients during their therapy sessions.

\section{Objectives}

\begin{itemize}
    \item The primary objective of this study is to investigate different position and orientation tracking algorithms and validate their
    \item The secondary objective is to develop an instrumented arm-skateboard (ArmBo)
    that can provide real-time feedback during training and assessing movement ability.
\end{itemize}

\section{Motivation}
\subsection{Objective 1}
Given the absence of current methods for effectively tracking the 
position and orientation of the upper limb without the use of costly robots, 
our aim was to explore various cost-effective alternatives for achieving the 
same goal, eliminating the need for expensive robotic technology.\\

\subsection{Objective 2}
The current arm-skateboards are simple, inexpensive, and easily made/procured, they lack movement sensing capabilities.
As a result, participants do not receive any feedback during training, which can lead to boredom and poor adherence in therapy.\\





\end{document}