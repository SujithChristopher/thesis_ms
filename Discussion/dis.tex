\documentclass[12pt, twoside]{report}
\usepackage{layout}
\usepackage[utf8]{inputenc}
\usepackage{graphicx}
\usepackage{amsmath}
\usepackage[a4paper, width=150mm, top=25mm, bottom=25mm]{geometry}
\usepackage{biblatex}
\addbibresource{bibs.bib}

\graphicspath{ {./src/} }


\begin{document}

\chapter{Discussion}


From the investigation of this study, we found that encoder based kinematic model
is not the best approach due to its limitations \parencite{lynch_modern_2017,li_kinematic_2020,taheri_kinematic_2015}. This is because of its inherent
nature of wheel slippage. \\

Even if the ArUco marker gives absolute posotion and orientation, the detection of the 
ArUco marker is not robust. This is because of the motion blur and occlusion.\\

However, we managed to address this issue by employing a 
deep learning technique. This proved to be an efficient method for 
determining the position and orientation of the Instrumented Arm-Skateboard (ArmBo) \parencite{zhang_deeptag_2021,garrido-jurado_automatic_2014,li_corner_2021}. \\

This allows us to track the position of the ArmBo in real-time, with a low resource 
hardware, instead of relying on a high-end GPU. This makes the ArmBo more affordable
and accessible to the patients. \\

To avoid marker occlution, we have used three ArUco markers. 
This allows us to track the ArmBo even if any two of the markers is occluded. \\

This simple setup can be used in both clinical and home settings.

\section{Future Work}

\begin{itemize}
    \item The ArmBo will be equiped with load cells in the arm rest. This will
    help us understand the amount of loading is given to the skateboard while performing
    the therapy. The arm rest will be spring actuated, as it will give the users
    haptic feedback while performing the therapy.
    \item We will be doing a usability study with healthy subjects, and improve the 
    usability of the ArmBo. Following which we will do a patient trial.
  \end{itemize}


\chapter{Conclusion}
In summary, we have created an affordable, open-source, and user-friendly arm skateboard, ArmBo, designed for arm motion therapy analysis. All components, including software, hardware (CAD models), and firmware, 
are open-source and can be easily customized to meet user requirements. \\


The algorithm, powered by the YOLOv8n mobile model, 
doesn’t necessitate a high-end GPU. It can operate at up to 33 Hz 
using only a CPU, making it compatible with low-resource devices like 
a Raspberry Pi 4 or a laptop. \\

ArmBo’s standout features include its extreme portability, 
the need for just a 720p webcam, and its quick setup time. 
These make it adaptable for use in diverse settings, from clinics to homes. 
It functions as an effective tool for patient arm movement analysis, 
providing real-time feedback. \\




\end{document}