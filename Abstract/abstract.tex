
\documentclass[12pt, twoside]{report}
\usepackage{layout}
\usepackage[utf8]{inputenc}
\usepackage{graphicx}
\usepackage{amsmath}
\usepackage[a4paper, width=150mm, top=25mm, bottom=25mm]{geometry}


\graphicspath{ {./src/} }

\begin{document}
\chapter*{Abstract}
\addcontentsline{toc}{chapter}{Abstract}
%page title

Stroke can cause motor function impairment in the arms and hands.
Arm skateboard is a mobility device that is commonly used to aid recovery from stroke.
Though conventionally used skateboards are simple, inexpensive, and easily made/procured, they lack movement sensing capabilities.
As a result, participants do not receive any feedback during training, which can lead to boredom and poor adherence in therapy.\\

The primary objective of this study is to investigate different position and orientation tracking algorithms and validate their
accuracy against a gold standard OptiTrack. The secondary objective is to develop an instrumented arm-skateboard (ArmBo)
that can provide real-time feedback during training and assessing movement ability. \\

The ArmBo platform consists of three omni-directional wheels and an arm-support on top.
It has two sensing modalities: optical encoders at the three omni-directional wheels, and an inertial measurement unit (IMU).
These sensing modalities can be used to estimate the position and orientation of ArmBo.
The platform is equipped with a Teensy 4.0 microcontroller that acquires real-time data from the optical encoders and the IMU.
The data is wirelessly streamed to a PC through an HC-05 Bluetooth module. \\

In our study, we investigated two distinct tracking algorithms to determine the position and orientation of ArmBo.
The first approach is a sensor-fusion based tracking method that uses an ArUco marker (detected using an RGB camera) and an IMU.
The second approach is an encoder-based tracking method. \\

Compared to the encoder-based approach, the sensor-fusion based approach yielded superior results
with a maximum absolute error of 4.1 cm, 0.2 cm, and 3.1 cm in the X, Y, and Z axes, respectively. \\

This in-house developed ArmBo provides a simple and effective way, to analyse arm motion therapy.\\

\end{document}